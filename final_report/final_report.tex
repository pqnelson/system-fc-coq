% Many thanks to Andrew West for writing most of this file
% Main LaTeX file for CIS400/401 Project Proposal Specification
%
% Once built and in PDF form this document outlines the format of a
% project proposal. However, in raw (.tex) form, we also try to
% comment on some basic LaTeX technique. This is not intended to be a
% LaTeX tutorial, instead just (1) a use-case thereof, and (2) a
% template for your own writing.

% Ordinarily we'd begin by specifying some broad document properties
% like font-size, page-size, margins, etc. -- We have done this (and
% much more) for you by creating a 'style file', which the
% 'documentclass' command references.
\documentclass{sig-alternate}
 
% These 'usepackage' commands are a way of importing additional LaTeX
% styles and formattings that aren't part of the 'standard library'
\usepackage{mdwlist}
\usepackage{url}
\usepackage{tabularx}
\usepackage{tikz}
\usetikzlibrary{shapes,arrows}
\usepackage[export]{adjustbox}
\usepackage{lipsum,adjustbox}
\usepackage{listings}% http://ctan.org/pkg/listings
\lstset{
  basicstyle=\ttfamily,
  mathescape
}\begin{document} 

% We setup the parameters to our title header before 'making' it. Note
% that your proposals should have actual titles, not the generic one
% we have here.
\title{Verification of System FC in Coq}
\subtitle{Dept. of CIS - Senior Design 2014-2015\thanks{Advisors: Stephanie Weirich (sweirich@cis.upenn.edu), Richard Eisenberg (eir@cis.upenn.edu).}}
\numberofauthors{4}
\author{
  Tiernan Garsys \\ \email{tgarsys@seas.upenn.edu} \\ Univ. of Pennsylvania \\ Philadelphia, PA\\\\
  Lucas Pe\~{n}a \\ \email{lpena@seas.upenn.edu} \\ Univ. of Pennsylvania \\ Philadelphia, PA
  \and
  Tayler Mandel \\ \email{tmandel@seas.upenn.edu} \\ Univ. of Pennsylvania \\ Philadelphia, PA\\\\
  Noam Zilberstein \\ \email{noamz@seas.upenn.edu} \\ Univ. of Pennsylvania \\ Philadelphia, PA
}
\date{}
\maketitle

% Next we write out our abstract -- generally a two paragraph maximum,
% executive summary of the motivation and contributions of the work.
\begin{abstract}
  \textit{
Haskell's compiler, the Glasgow Haskell Compiler (GHC), generates code in GHC Core. The Coq proof assistant is used to verify substantial subset of System FC, the theoretical basis for GHC Core. A translation from the formal language to GHC Core, the concrete implementation of System FC that is used in GHC, will then be proven. The goal of verification is to prove that the evaluation semantics of System FC are sound.
  }

  \textit{
There are two main benefits to this project. First, the verification would provide assurance regarding the safety and accuracy of GHC. Second, and perhaps more importantly, it will provide foundation to verify other properties of GHC such as compiler optimizations.
 }

\textit{
Haskell is a statically-typed functional programming language that is commonly used for the compile-time guarantees about program behavior that its type system is believed to provide. Despite this usage, the type safety of  Haskell has not been formally proven. A foundation for a mechanized proof of Haskell's type safety is presented via a mechanized proof of the type safety of a substantial subset of System FC, the theor
 }
\end{abstract}

% Then we proceed into the body of the report itself. The effect of
% the 'section' command is obvious, but also notice 'label'. Its good
% practice to label every (sub)-section, graph, equation etc. -- this
% gives us a way to dynamically reference it later in the text via the
% 'ref' command, e.g., instead of writing `Section 1', you can write
% `Section~\ref{sec:intro}', which is useful if the section number
% changes.

%  
%
%
\section{Introduction}
\label{sec:intro}


Content content content

\section{Background}
\label{sec:background}

\section{Motivation}
\label{sec:motivation}

Content content content

\section{Related Work}
\label{sec:related-work}

Content content content

\section{Implementation}
\label{sec:implementation}

Content content content

\section{Results}
\label{sec:results}

Content content content

\section{Future Work}
\label{sec:future-work}

Content content content

\section{Ethical Considerations}
\label{sec:ethics}

lol

\section{Conclusion}
\label{sec:conclusion}

Content content content

% We next move onto the bibliography.
\bibliographystyle{plain} % Please do not change the bib-style
\bibliography{final_report}  % Just the *.BIB filename

% Here is a dirty hack. We insert so much vertical space that the
% appendices, which want to begin in the left colunm underneath
% "references", are pushed over to the right-hand column. If we looked
% hard enough, there is probably a command to do exactly this (and
% wouldn't need tweaked after edits).
\vspace{175pt}

\end{document} 
